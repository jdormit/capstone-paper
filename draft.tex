\documentclass[titlepage]{article}

\title{Open-Source Knowledge: Information Exchange and Copyright in the Digital Era}
\author{Jeremy Dormitzer}
\date{April 28, 2017}

\begin{document}

\maketitle

\section{Introduction}\label{introduction}

Digital computing has fundamentally changed the way that our society
creates, digests, and exchanges knowledge. Although these changes
increase access to information, enable global discourse, and connect
minds on opposite sides of the world, they also have troubling power and
privacy implications. These implications are not widely discussed in
popular literature. However, some scholars and activists have long been
aware of the dangers of knowledge exchange in the digital age. In his
1979 work ``The Postmodern Condition: A Report on Knowledge''
(specifically chapter one: ``The Field: Knowledge in Computerized
Societies''), the French philosopher Jean-François Lyotard predicted a
shift in the status of knowledge. He believed that the advent of
computers would cause knowledge to become commoditized, a unit of
exchange rather than something with inherent value. Furthermore, he
detailed the problematic political and economic consequences of this
shift. As we will see, many of his predictions have already come to
pass.

More recently, software developer and privacy activist Richard Stallman
has articulated many similar concerns. Although Stallman primarily talks
and writes about software, the concerns of his philosophy closely
parallel those of Lyotard's book. Stallman believes that the idea that
software can have an owner leads to invasive privacy breaches and a
growing power imbalance in which business interests trump those of the
greater good of society. Stallman advocates for a simple but powerful
solution to this problem: software should not have owners. He offers a
compelling case for ``free software'' (Stallman explains this concept
with the phrase, ``free as in freedom, not free as in beer''), and in
his many essays and talks he demonstrates why free software solves the
privacy and power concerns he sees in the modern world.

In this paper, I endeavor to show how Stallman's ``free software''
philosophy can be used to address Lyotard's concerns about knowledge in
the digital era. I begin with an analysis of Lyotard's concerns for the
changing state of knowledge, tracing his reasoning from its foundations
to its consequences. Next, I perform a similar analysis of Stallman's
problem, and explain how free software solves it. I show that Stallman's
conception of software plays the same role in his work as does knowledge
in Lyotard's. This sets the stage for the body of the paper, in which I
extend Stallman's reasoning about software to the field of knowledge in
general. I show that, when generalized, Stallman's ``free software''
philosophy offers hope that Lyotard's concerns could be mitigated. I
conclude with an exploration of the implications of ``free knowledge''.

\section{Lyotard's Prediction}\label{lyotards-prediction}

In order to understand Lyotard's predictions about the changing status
of knowledge, we must first investigate his conception of knowledge. He
begins by narrowing his topic of discussion to scientific knowledge
only; this allows him to get specific about how and why knowledge is
changing. Having settled on a topic, Lyotard gives a preliminary
definition of knowledge: ``Scientific knowledge is a kind of discourse''
(Lyotard 299). For Lyotard, then, knowledge is not simply a unit,
something that one has or does not have; instead, knowledge is a
conversation, a shifting shared entity that changes over time as point
leads to counterpoint. Lyotard contends that - at the time he was
writing - ``for the last forty years the `leading' sciences and
technologies have had to do with language'' (299). Language, in this
context, is taken very generally: Lyotard cites not only sciences of
linguistics, communication, and translations, but also algebra, data,
and computer programming. Language sciences, in other words, are those
that deal with manipulating symbols of meaning. This shift towards
language sciences is the first sign of the change in the status of
knowledge, and lead to the change's symptoms: first, a tendency in other
fields of research to draw on language sciences; second, and crucially,
an increased use of ``information processing machines'' (300) to
circulate knowledge on a broader scale than was previously possible.
Lyotard supposes that this latter change will have ``as much of an
effect on the circulation of learning as did advancements in human
circulation (transportation systems), and later, in the circulation of
sounds and visual images (the media)'' (300). In this, as we will see,
Lyotard has proven to be prescient.

Having understood the type of knowledge that Lyotard writes about, as
well as a broad overview of the change in the status of this knowledge
that will come about, we can investigate the specific concerns that
Lyotard raises about this transition. He writes, ``We may thus expect a
thorough exteriorisation of knowledge of knowledge with respect to the
`knower'\,'' (300). What does this mean? Lyotard draws an analogy to
industry: ``The relationships of the suppliers and users of knowledge to
the knowledge they supply and use is now tending {[}\ldots{}{]} to
assume the form already taken by the relationship of commodity producers
and consumers to the commodities they produce and consume - that is, the
form of value'' (300). In the face of new language sciences, knowledge
ceases to be a conversation. Instead, it becomes a commodity. Why does
this come about? In short, because of computers. Computers understand
only data, so all knowledge that passes through a computer must be in
the form of data. Lyotard predicts that ``anything in the constituted
body of knowledge that is not translatable {[}to data{]} will be
abandoned and that the direction of new research will be dictated by the
possibility of its eventual results being translatable into computer
language'' (300). This necessity means that knowledge must become
discrete instead of continuous. One can speak of this piece of knowledge
or that piece of knowledge, and each piece has physical limits defined
by its file size or data type. Knowledge becomes an object - one that,
like all objects, can be owned, bought, and sold.

Lyotard explains that the commoditization of knowledge will have broad
societal, political, and economic consequences. First, as a commodity,
knowledge is now a thing that can be produced and exported by a
nation-state; indeed, Lyotard writes that ``knowledge has become the
principal force of production over the last few decades; {[}\ldots{}{]}
In the postindustrial and postmodern age, science will maintain and no
doubt strengthen its preeminence in the arsenal of productive capacities
of the nation-states'' (300). This shift has a noticeable effect on a
nation's workforce. It creates an increasingly important and lucrative
``white-collar'' job market. Knowledge workers make more money than
laborers, and this in turn bolsters the nation's economy, creating a
positive feedback loop in which a better economy means more tech
companies, and more tech companies means more knowledge workers, which
leads to an even better economy. However, in order to kick off this
process there need to be knowledge workers and tech companies already
working in that nation; this means that developing nations without such
resources get left further and further behind.

Second, Lyotard points out that commoditized knowledge becomes a scarce
resource over which nations and businesses must compete. He writes,
``Knowledge in the form of an informational commodity indispensable to
productive power is already, and will continue to be, a major
{[}\ldots{}{]} stake in the worldwide competition for power'' (301). In
other words, knowledge resources both enable and are themselves a force
of production, and therefore a source of power. Nation-states will need
to develop new industrial, economic, and military policies to succeed in
this new world.

Third, Lyotard predicts that knowledge's new status will upset the power
balance between the public sector and the private sector. Traditionally,
the state controlled the ``production and distribution of learning''
(301). When talking about learning, Lyotard means more than simply going
to a university or reading a book; he means control over the
transmission of knowledge, from the individual level all the way up to
determining the \emph{ethos} of a generation. Before the commoditization
of knowledge, the state enjoyed regulatory power over the majority of
knowledge discourse. However, as knowledge becomes a product, control
over the contents of that product moves to the product's owner -
usually, a private business entity. In particular, Lyotard is concerned
about the rise of multi-national corporations - businesses so large and
so distributed they do not fall under the jurisdiction of any one
nation-state. He gives a pertinent example:

\begin{quote}
Suppose, for example, that a firm such as IBM is authorised to occupy a
belt in the earth's orbital field and launch communications satellites
or satellites housing data banks. Who will have access to them? Who will
determine which channels or data are forbidden? The State? Or will the
State simply be one user among others? New legal issues will be raised,
and with them the question: `who will know?' (301)
\end{quote}

Lyotard demonstrates the full frightening implication of knowledge's new
status: the entity that controls the channels of knowledge controls the
knowledge itself. That entity therefore has power over everyone who
wants or needs that knowledge. The questions that Lyotard asks about the
communications satellites concern every vehicle of knowledge: questions
of access control, censorship, privacy. Although Lyotard wrote in 1979,
these questions have become increasingly relevant. Many of Lyotard's
predictions about the status of knowledge and its consequences have come
to pass. Richard Stallman, in his many writings and talks, has taken up
Lyotard's banner and continues to raise concerns about the breadth and
scope of the power of those who control the knowledge. Furthermore, as
we will see, Stallman offers hope that that power could be redistributed
back to those to whom it rightfully belongs: the people themselves.

\section{Stallman's Problem}\label{stallmans-problem}

Richard Stallman, from about a decade after Lyotard published his grim
predictions to the present, writes and gives talks about a problem he
sees with modern copyright law. Although he mostly discusses software,
his arguments apply to anything covered under copyright law, and as we
will see, actually apply to any form of knowledge as defined by Lyotard.
His writing is prolific; however, his views on copyright law and freedom
are neatly summarized in a speech he gave in 2001 entitled ``Copyright
and Globalization in the Age of Computer Networks''. In that talk,
Stallman argues that his views on free software apply to any kind of
information that can be stored on a computer; therefore, in order to
understand his arguments there we will first need to understand his
views on free software. Stallman lays these views out neatly in an
earlier essay, ``Why Software Should Be Free''.

In ``Why Software Should Be Free'', Stallman explains that the principle
concern of free software is that of ownership. He asks, ``suppose on
individual who has a copy of a program meets another who would like a
copy. It is possible to copy the program; who should decide whether this
is done? The individuals involved? Or another party, called the
`owner'?'' (Stallman 121). Stallman holds that the usual criteria for
answering this question is that which maximizes developers' profits.
Based on that criteria, the question is answered with the idea that the
software does have an owner separate from the user, and this owner gets
to decide who uses the software. This is the typical answer, and the
answer that most people today accept as true. However, Stallman proposes
a different criteria by which to answer the question: ``the prosperity
and freedom of the public in general'' (121). By this criteria, he
argues, software should not have owners. This is what he means by the
phrase ``free software'': software without an owner.

Why does Stallman believe that software without an owner is better for
society? Software ownership implies restrictions on the use of that
software: under copyright protection, a user is not allowed to modify or
copy the program they are using. Stallman compares this model to a road
with a toll booth. Drivers on the toll road would have to pay to use it,
just as users of software must pay to use it. This payment would fund
the development of the road. However, most roads are not toll roads.
Stallman writes, ``Comparing free roads and toll roads by their
usefulness, we find that (all else being equal) roads without toll
booths are cheaper to construct, cheaper to run, safer, and more
efficient to use. {[}\ldots{}{]} The toll road is not as good as the
free road'' (123). In the same way, software with owners is ``more
expensive to construct, more expensive to distribute, and less
satisfying and more efficient to use'' (123). Stallman gives three
reasons why this is the case:

\begin{quote}
\begin{enumerate}
\def\labelenumi{\arabic{enumi}.}
\item
  Fewer people use the program.
\item
  None of the users can adapt of fix the program.
\item
  Other developers cannot learn from the program, or base new work on
  it. (124)
\end{enumerate}
\end{quote}

These harmful effects reduce the value of the software. However, they
are not the only harm which occurs from software ownership. Stallman
argues that the damage to society from software ownership far outweighs
the damage to the value of individual programs. This societal damage
occurs because of what he calls ``psychosocial harm'', that is, ``the
effect that people's decisions have on their subsequent feelings,
attitudes, and predispositions'' (124). The decisions that software
ownership forces people to make ends up changing their worldview for the
worse. If everyone makes those types of decisions, it changes everyone's
worldview for the worse. Stallman gives many specific examples of
decisions that change someone's worldview. He sketches a scenario in
which a program user feels a social obligation to help his neighbor by
sharing the program with the neighbor. However, copyright law makes this
sharing illegal. Stallman argues that this tension does psychosocial
damage to the user by forcing them to suppress their natural instinct to
help. On the other side of the equation, the developer of the program
suffers psychosocial harm due to the discouraging feeling of knowing
that not everyone will be permitted to use the software she develops.

Although these examples seem somewhat contrived, reflection reveals that
these types of situation appear quite frequently in modern society, and
that the negative affects associated with them really do occur. In a
2015 Wired Magazine article entitled ``We Can't Let John Deere Destroy
the Very Idea of Ownership'', reporting on a comment the tractor company
submitted to the Copyright Office regarding the ownership status of the
software in their tractors, writer Kyle Wiens reveals the harm done by
proprietary software: ``{[}farmer{]} Kerry Adams {[}\ldots{}{]} hasn't
been able to fix an expensive transplanter because he doesn't have
access to the diagnostic software he needs'' (Wiens 1). This restriction
commits a real social harm against farmers. Another farmer told Wiens,
``We are used to operating independently, and that's one of the great
things about being a farmer. And in this particular space, they are
really taking that away from us'' (1). Software ownership has real,
profound consequences on the social well-being of software users.

Lyotard's arguments, of course, applied to a much broader field than
just software, and in order to be useful to us we must show that
Stallman's ideas do as well. In ``Copyright and Globalization in the Age
of Computer Networks'', Stallman broadens his argument for free software
into an argument against copyright for all types of information. He
asks, ``how do the ideas about freedom for software users generalize to
other types of things?'' (Stallman 135). There are only certain other
types of things that free software ideas can apply to. Specifically,
Stallman argues that these ideas can be applied to certain types of
information: ``{[}A{]}ny kind of information that can be stored on a
computer, conceivably, can be copied and modified. So, the ethical
issues of free software, the issues of a user's right to copy and modify
software, are the same as such questions for other kinds of published
information'' (136). The phrase ``published information'' is very
important here. Although there are many types of information that can be
copied and modified, ethical arguments about the freedom of that
information only apply if the information is published, that is, made
publicly available. Stallman's argument, therefore, is one against the
system of copyright in general, since that is the system that prevents
copying and modification of published information.

To contextualize his arguments about the harmful effects of copyright,
Stallman presents a summary of the history of copyright. Copyright, he
argues, used to be a socially useful construct. Before digital
technology, information has to be either copied by hand or copied on a
printing press. It was expensive to own and operate a printing press;
therefore, copyright was really a regulation on businesses that could
afford printing presses rather than on the general public. It protected
authors and publishers from other businesses stealing their work. It was
easy to enforce, since any business that copied a book would need to
publicly share that copying in order to profit from it. Users of the
information, i.e.~those that purchased the printed copies, were free to
make copies by hand, since hand-copying was no threat to the publishers'
businesses. Stallman argues that this early form of copyright was
actually beneficial for the public: if information publishers know they
can profit from making the information, more information will get made
in general, and the public has not lost anything since they did not have
a good way to copy the information anyway.

However, computers have changed the equation. Stallman writes,
``Computer networks and digital information technology are bringing us
back to a world more like the ancient world, where anyone who can read
and use the information can also copy it and can make copies about as
easily as anyone else could make'' (138). Although copyright laws still
exist to protect the publishers, the laws no longer protect publishers
against other publishers. Now, they protect publishers against the users
- against the public.

When applied to a greater field than just software, the troubling
effects of information ownership become clearer. If owners need to keep
information from being copied or modified by users, then they need to
keep an eye on all their users to make sure none of them break the
copyright: ``To enforce {[}copyright{]} requires surveillance, an
intrusion, and harsh punishments'' (138). This surveillance can take the
form of computer programs that report back to their owners who is using
them, or restrictions on the copying of data using various technologies
collectively called `digital rights management', abbreviated to `DRM'.
Furthermore, if publishers retain control over all the information they
publish, they can repeal that information at any time. Stallman cites
the example of a protest site criticizing a major bank: the site was
simply disconnected by its ISP (internet service provider) without ever
seeing the inside of a courtroom. It did not own the channels through
which its data flowed, and so was powerless to do anything about the
situation.

Viewed in this light, the state of modern copyright law seems grim. But
Stallman offers a solution. Just as his solution to the problem of
software ownership is free software, that is, software that can be
copied or modified without restriction, Stallman's solution to the
problem with copyright law in general is to decrease the restrictions it
imposes. However, he does not advocate abolishing it entirely. He
writes, ``instead of increasing copyright powers, we have to pull them
back so as to give the general public a certain domain of freedom where
they can make use of the benefits of digital technology, make use of
their computer networks'' (143). So how far should the restrictions
imposed by copyright be repealed? The answer, according to Stallman,
depends upon the type of information being protected by copyright.

Stallman divides the realm of published information into three kinds of
works: functional works, works intended to say what certain people
think, and aesthetic works. Aesthetic works are those created for their
artistic merit: music, novels, etc. Stallman views their copyright
status as complicated, since on the one hand modifying them would change
their author's artistic vision, but on the other hand much great art is
created by adapting the works of others. Stallman's second category of
works, those intended to convey what someone thinks, include memoirs,
speeches, blogs, news articles, and other such publications. These types
of works should be legal to copy but not to modify, according the
Stallman, since modification would misrepresent the intentions of the
authors. Functional works are those ``whose use is to get a job done''
(143). Stallman explains that functional works should be totally free,
able to be both copied and modified. This is because of their utility.
Stallman says, ``People should have the freedom even to publish a
modified version because it's very useful to modify functional works''
(143). We can see an real-world example of this in the open source
software movement. Developers allowing others to copy and modify the
source code of many types of software has led to an explosion of useful
utilities, to the point that the majority of cloud servers run on the
open-source operating system Linux. This lines up with Stallman's
criteria for what will be best for society - allowing people to modify
functional works leads to the greatest prosperity and freedom for the
public in general.

Of the three types, functional works are the most relevant to this
discussion. As we will see, functional works bear the most resemblance
to Lyotard's conception of knowledge, and so Stallman's conclusions
regarding them can be applied more broadly to address Lyotard's
concerns. In order to prove that claim, however, we must first show that
Stallman's and Lyotard's concerns actually coincide.

\section{Applying Stallman to
Lyotard}\label{applying-stallman-to-lyotard}

Analysis reveals that Lyotard's conception of knowledge is of the same
category as Stallman's conception of software, and published information
more generally. To see this, we must return to Lyotard's definition of
knowledge. As we discovered, Lyotard believed that (scientific)
knowledge is a discourse, a conversation, but the advent of digital
information technologies discretised that knowledge into a bounded
object, a commodity that could be bought and sold. We can find many
points of similarity between this description and the language that
Stallman uses to talk about copyright.

In Stallman's historical analysis of copyright, he makes an important
point about the status of copying information before the digital age. He
says, ``there wasn't, in the ancient world, the same total divide
between copying a book and writing a book. {[}\ldots{}{]} For instance,
you could copy a part of a book, then write some new words, copy some
more and write some new words, and on and on. This was called `writing a
commentary'\,'' (136). He also cites writing a compendium, in which bits
from many different works were copied, as an example of something useful
between copying and writing. This corresponds to Lyotard's conception of
knowledge as a conversation: works by one author were adapted and
responded to by another, and those secondary works themselves had
responses and offshoots. The net result of this process was that society
benefited, since the most useful ideas would get copied the most.

What about the transition that Lyotard wrote about? He cited the advent
of computers as changing the nature of knowledge, since knowledge now
had to be represented as a discrete quantity that could be stored as
data. Although Stallman does not explicitly document this change, the
limits he places on his topic imply it. Stallman claims that his
argument about copyright applies to ``any kind of information that can
be stored on a computer'' (136). This type of information, by
definition, must be of the same type as the knowledge that Lyotard
writes about.

Lyotard expresses a concern that knowledge will become a commodity -
something that is produced in order to be sold, and something that is
bought as a form of capital to be used in further production rather than
for its own sake. We can find echos of this sentiment in Stallman's
criticism of copyright law. At its core, Stallman's argument concerns
who controls information, and what level of control they have over it.
If we take Lyotard's analogy to its logical conclusion, we find a world
in which the only way to obtain knowledge is to buy it, and once bought,
the only goal is to use it to produce more knowledge to sell. This bears
a more-than-passing resemblance to the role occupied by information
publishers in Stallman's work - specifically, publishers of so-called
functional works. As an example, take a company that sells soda. This
company has a recipe they use to make their soda, and only they know
what it is. The ingredients for soda are common and easy to obtain, so
customers of this company are not buying the water and the sugar and the
fruit juice that goes into the soda they buy. Rather, they are buying
the secret configuration of all those ingredients, the copyrighted
information that reveals how the soda is made. In that sense, the soda
company is not a soda company at all. They sell information. When the
soda company was founded, perhaps its owner bought that information from
a previous company. If a former employee wanted to start their own soda
company, they would need to either invent their own soda recipe - or buy
the soda company's recipe. The soda company, stripped of its
industry-specific trappings, is a company that buys information in order
to produce new information.

Stallman describes a world in which Lyotard's prediction has come to
pass. However, unlike Lyotard, he offers a solution to the problems he
raises. Those answers also address the broad issues that Lyotard fears:
an increased income gap between developing countries and first-world
countries, and a dangerous concentration of power in corporations as
they gain control over the world's knowledge supply. Stallman's proposes
amending copyright laws, specifically for functional works, to make them
free - free as in freedom, not as in beer. This neatly solves Stallman's
concerns: if a work is free to copy and modify, then all of society
benefits, draconian surveillance is not necessary, and the interests of
the public would once again trump the interests of private businesses.

How does Stallman's solution address Lyotard's more broad concern? The
answer can be found by considering the question that Lyotard raises in
his IBM example - who will know? That question is the root of Lyotard's
concerns about gaps between countries and concentrated, unregulated
corporate power. Who will know? If the answer is, ``whoever owns the
knowledge or has the resources to buy it'', then Lyotard's fears will
come to pass. As Stallman has shown us, that is often the case in the
modern world. However, if the knowledge in question is free - free to
copy and free to modify - then the answer to ``who will know'' is
``anyone who wants to''. In effect, free knowledge is knowledge that has
ceased to be commoditized. Knowledge that can be copied and modified can
be responded to, reproduced, rebutted. People can write a commentary
about it, or put it in a compendium. Developing countries can catch up
to first-world countries, because the knowledge economy that currently
holds them down would be free for the taking. Corporations could not
hoard knowledge to consolidate their power. Free knowledge is once again
a discourse.

\section{Conclusion}\label{conclusion}

Although Lyotard and Stallman both paint dark pictures of the state of
knowledge in the digital era, Stallman offers a solution that mitigates
both his and Lyotard's concerns. By generalizing Stallman's ``free
software'' philosophy, drawing on his proposed changes to copyright law,
and applying them to Lyotard's description of knowledge, we can see a
way in which knowledge continues to be a conversation, one that benefits
all of society instead of lining the pockets of private corporations.

Of course, the major question that remains is ``how can we do this in
practice?''. Stallman offers some suggestions. The authors of
information need to make a living, and a common objection to Stallman's
ideas are that authors will not get paid if they are not protected by
copyright. To prevent this, Stallman suggests selling support contracts,
soliciting donations, and forming non-profits to obtain grant money. In
addition he points out that free knowledge does not necessarily have no
price - one can imagine selling packages that contain free software, or
selling works that retain a free license. Of course, if the work is free
to be copied and modified, nothing forces people to pay for it, and this
is an unsolved issue that still bars adoption of Stallman's ideas.

Other issues rear their heads when considering Stallman's ideas.
Copyright was implemented for a good reason - it provides an incentive
for people to produce information. Other incentives could be conceived
of to produce knowledge, of course, but would those incentives be as
strong? Or would eliminating copyright for functional works reduce the
total volume of those types of works produced? Additionally, although
the current copyright system invites abuse by corporations and others in
power, a more open system could invite other forms of abuse, making it
easier for the unscrupulous to plagiarize and steal.

Despite these barriers, the potential of free knowledge offers many
opportunities to society. Imagine a totally free education system, in
which textbooks collaboratively written by the subject's finest teachers
were freely available. Imagine a research system whose results were
available to everyone, whether the researchers were academic, corporate,
or simply avid hobbyists. Imagine farmers in developing nations
introducing open source, specially bread strains of wheat that could
bring food to famine stricken regions. If everyone can know, everyone
will learn and everyone will teach, and the whole of society will be
better off.

\begin{thebibliography}{9}

\bibitem{lyotard84}
Lyotard, Jean-François.
\emph{The Postmodern Condition: A Report on Knowledge}.
Manchester, UK: Manchester University Press,
1979/1984.

\bibitem{stallman02}
Stallman, Richard M.
\emph{Free Software, Free Society: Selected Essays of Richard M. Stallman}.
Ed. Joshua Gay.
Boston, MA: GNU Press, 2002.

\bibitem{wiens15}
Wiens, Kyle.
``We Can't Let John Deere Destroy the Very Idea of Ownership.''
\emph{Wired}.
Conde Nast, 21 April 2015.
Web.

\end{thebibliography}

\end{document}